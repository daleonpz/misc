\documentclass[]{llncs}

\usepackage[T1]{fontenc}
\usepackage{lmodern}
\usepackage{amssymb,amsmath}
\usepackage{ifxetex,ifluatex}
\usepackage{fixltx2e} % provides \textsubscript
% use upquote if available, for straight quotes in verbatim environments

% lncs 
\usepackage{makeidx}
\institute{FH Dortmund, \\ \texttt{}}


\IfFileExists{upquote.sty}{\usepackage{upquote}}{}
\ifnum 0\ifxetex 1\fi\ifluatex 1\fi=0 % if pdftex
  \usepackage[utf8]{inputenc}
\else % if luatex or xelatex
  \ifxetex
    \usepackage{mathspec}
    \usepackage{xltxtra,xunicode}
  \else
    \usepackage{fontspec}
  \fi
  \defaultfontfeatures{Mapping=tex-text,Scale=MatchLowercase}
  \newcommand{\euro}{€}
\fi
% use microtype if available
\IfFileExists{microtype.sty}{\usepackage{microtype}}{}
\usepackage{longtable,booktabs}
\ifxetex
  \usepackage[setpagesize=false, % page size defined by xetex
              unicode=false, % unicode breaks when used with xetex
              xetex]{hyperref}
\else
  \usepackage[unicode=true]{hyperref}
\fi
\hypersetup{breaklinks=true,
            bookmarks=true,
            pdfauthor={Daniel Paredes},
            pdftitle={I need a title},
            colorlinks=true,
            citecolor=blue,
            urlcolor=blue,
            linkcolor=magenta,
            pdfborder={0 0 0}}
\urlstyle{same}  % don't use monospace font for urls
\setlength{\parindent}{0pt}
\setlength{\parskip}{6pt plus 2pt minus 1pt}
\setlength{\emergencystretch}{3em}  % prevent overfull lines
\setcounter{secnumdepth}{5}

\title{I need a title}
\author{Daniel Paredes}
\date{}

\begin{document}
\maketitle
\begin{abstract}
I need an abstract

% lncs keyword extension 
\keywords{I, need, keywords}

\end{abstract}

\hypertarget{introduction}{%
\section{Introduction}\label{introduction}}

I need an introduction\ldots{}

\hypertarget{my-approach-i-need-a-section-title}{%
\section{My approach (i need a section
title)}\label{my-approach-i-need-a-section-title}}

\hypertarget{tacit-knowledge-in-elicitation-interviews}{%
\subsection{Tacit knowledge in elicitation
interviews}\label{tacit-knowledge-in-elicitation-interviews}}

Requirements elicitation is the process of finding and formulation
requirements, and this process has many barriers due to stakeholders are
not able to express their needs, explain what they do and why,
conflicting demands, or new demands once other are met{[}1{]}, just to
mention some examples. Many analyst consider interviews the most
important elicitation technique, and usually elicitation process starts
with interviewing the stakeholders. Interviews allow analysts to check
their understanding about the problem domain immediately and ask for
clarifications. Moreover, this technique is versatile in the sense of it
is possible that new and unexpected issues come up and they can be
attacked at that moment.

However, elicit tacit knowledge in interviews is still a hard task to
do. This tacit knowledge can be classified in four groups as describe in
{[}2{]}. The \textbf{known knowns}, \textbf{known unknowns},
\textbf{unknown knowns} and \textbf{unknown unknowns}. The known knowns
is the knowledge that is documented, expressible, and relevant to the
system. In order words, is the scenario in which it is possible to
elicit since the beginning all the requirements, there is no tacit
knowledge. The known unknowns is the knowledge that cannot be
expressible because the stakeholders are not aware of their lack of
domain knowledge, their might forget it. In this case analyst are aware
of the domain knowledge and therefore, their job is to challenge
assumptions or implications. The unknown knowns is the knowledge that
stakeholders hold but it is not documented by some reason. Thus,
analysts should use their interviewing skill to elicit when they spot
any glimpse of that knowledge. The unknown unknowns is the knowledge
that neither the analysts nor stakeholders are aware.

\hypertarget{challenge-tacit-knowledge}{%
\subsection{Challenge tacit knowledge}\label{challenge-tacit-knowledge}}

The unknowness clarification presented in above suggest the presence of
three problems. The goal is to go from any scenario in which tacit
knowledge exists to known knowns. In order to achieve that goal and take
the advantages presented in elicitation interviews (immediately
clarification of assumptions and debate of implications) we are going to
use the theoretical framework described by Elrakaiby {[}3{]}. In this
framework we model the interview as a set of arguments and attacks
between them. In that sense, we will able to spot ambiguities and tacit
knowledge in terms of inconsistencies and insufficiency of information.
We will adapt some examples from {[}3{]} to show how each tacit
knowledge scenario can be addresed.

\textbf{known unknowns}: In the following example the domain knowledge
of the analyst is used to detect inconsistencies (attacks), because the
stakeholder didn't take into account relevant information about domain
application (location of trash bins).

A customers wanted a recycling-support system and one of the goals was
\emph{to avoid fines from the municipality for incorrect recycling}.
According to the domain knowledge of the analyst, \emph{trash bins are
placed along the streets}. Thus, the goal was inconsistent with the
domain knowledge of the analyst because \emph{there is no way to trace
back person who brakes the recycling rule}. This scenario can be modeled
using arguments and attacks as follows:

The arguments: (A1)trash bins are placed along the streets, (A2)since
trash bins are placed along the streets, the trash owner cannot be
traced back. (B1)people who do not recycle should be fined (B2)to fine
people, the municipality must be able to trace products in trash bins
back to their owners.

Attacks: A2 attacks B1, and viceversa.

\textbf{unknown knowns}: In the following example analyst challenges the
stakeholder statement, as a result the stakeholder reveals new
information (reply).

A customers wants to develop a system to allow patients to measure the
amount of glucose in their blood, and then send the result to their
general practitioner. If the glucose level is above a certain threshold,
the practitioner pays a visit to the patient. The customer said:
\emph{in one-two days the doctor sees the notification in the system}.
Then analyst asked: \emph{What happens if the doctor is not available
for any reason?}. The customer replied: \emph{The general practitioner
is substituted by another doctor who accesses the same system}.

The arguments: (A1)in one-two days the doctor sees the
notification,(B1)the doctor might be unavailable,(B2)since the the
doctor might be unavailable, the notification might be delayed.

The attacks: A1 attacks B2, and viceversa

\textbf{unknown unknowns}: In the following example the stakeholder
didn't notice the inconsistency between his statements, but the analyst
did and asked for clarification. However, the stakeholder couldn't give
an answers due to his ignorance.

One of the customers wants to develop an app to manage medical-related
reservation in Tuscani. The costumer mentioned that the \emph{current
reservation system, based on phone calls, was centralised}. During the
interview, the customer said that \emph{depending on where the
examination/visit will be, the patient has to call Nottola or Siena}.
The analyst could not understand how this realization was possible since
it means the reservation system is not centralized. Therefore the
analyst asked for clarification and the stakeholder couldn't give any
acceptable answer.

The arguments: (A1)the current reservation is centralized, (B1)depending
on where the examination/visit will be (B2)the patience has to call
Nottola or Siena, (B3) since it is possible to call either Nottola or
Siena the system is not centralized.

The attacks: A1 attacks B3, and viceversa

\hypertarget{limits-of-elicited-knowledge}{%
\subsection{Limits of elicited
knowledge}\label{limits-of-elicited-knowledge}}

In the last section we present how to use a theoretical framework based
on arguments and attacks to addresed the different levels of tacit
knowledge. In this section we will compare in which scenario of tacit
knowledge the elicitation interview is more effective based on Aranda et
al.~work {[}4{]} about the effect of domain knowledge on elicitation
process.\\
Effectiveness is described in {[}4{]} as the total percentage of problem
domain elements identified compared to a benchmark. However, since we
have no accesse to any requirement data, we will use the conclusions of
{[}4{]} to compare the different scenarios of tacit knowledge (known
unknowns, unknown knowns and unknowns unknowns).

The works presented in {[}4{]} concludes that: (1)analyst's domain
knowledge(ADK) has small but statisticall significant effect on the
effectiveness, (2)stakeholders' domain knowledge(SDK) has a big positive
and significant influence and (3)analyst's interviewing skill and
general training in requirement activities have also a big positive and
significant influence. In this paper, we will focus only in comparing
based on the domain knowledge holder since that knowledge can be
measured.

\begin{longtable}[]{@{}lcc@{}}
\caption{Influence of domain knowledge in the different tacit knowledge
scenarios}\tabularnewline
\toprule
scenario & ADK & SDK\tabularnewline
\midrule
\endfirsthead
\toprule
scenario & ADK & SDK\tabularnewline
\midrule
\endhead
known unknowns & \(+\) & 0\tabularnewline
unknown knowns & 0 & \(++\)\tabularnewline
unknown unknowns & 0 & 0\tabularnewline
\bottomrule
\end{longtable}

We can observed in Table 1 that it is more likely to be effective in the
scenario of \emph{unknown knowns} because the stakeholders' domain
knowledge have more influence than analyst's domain knowledge. On the
other hand, the analyst's interviewing skills will have different
approaches in the case of \emph{known unknowns} and \emph{unknown
knowns}. In the first case, the interview will be more focused on
challenging assumptions and implications, and in the latter case on
spoting glympses of stakeholders' hidden knowledge.

\hypertarget{papers-summary}{%
\section{Papers summary}\label{papers-summary}}

\hypertarget{requirements-elicitation-towards-the-unknown-unknowns-sutcliffe2013requirements}{%
\subsection{Requirements elicitation: Towards the Unknown Unknowns
{[}2{]}}\label{requirements-elicitation-towards-the-unknown-unknowns-sutcliffe2013requirements}}

Elicitation in requirements engineering is still problematic because
missing or mistaken requirements are hard to elicit, and this produce
projects delays and have financial implications. In this paper, the
authors propose an elicitation review framework (ERF) in order to
explore the different challenges related to the ``unknowness'' of the
domain knowledge. Therefore, proposing a road map of research to tackle
the different levels of the ``unknowness''.

The authors made use of the properties of \emph{expressible} (known
knowledge), \emph{articulated} (documented domain knowledge),
\emph{accessible} (need memory recall), \emph{relevant} (to the
project). With these four properties, authors defined the following
levels of unknowness in domain knowledge.

\begin{itemize}
\tightlist
\item
  Known knowns: knowledge that is expressible, articulated, and
  relevant.
\item
  Known unknowns: knowledge that is not expressible or articulated, but
  accessible and potentially relevant.
\item
  Unknown knowns: knowledge that is potentially accessible but not
  articulated.
\item
  Unknown unknowns: knowledge that is not expressible, articulated nor
  accessible but may be relevant.
\end{itemize}

This levels of knowledge implies problems of different perspective of
the elicitation process that involves analysts and stakeholders. The
\emph{known knowns} is a simple scenario since the counterparts are
aware of the domain problem which is expressible and articulated. In the
case of the \emph{known unknowns}, the analysts has a domain knowledge
and they should be able draw out the information from the stakeholders
that they might be unaware of. In contrast, in the \emph{unknown knowns}
scenario the stakeholders are the ones that are aware of the domain
knowledge but for some reason it is no articulated, and therefore, the
analysts once they notice any glimpse of the stakeholders information
they should able to discover and elicit that. On the other hand, in the
case of \emph{unknown unknowns} both counterparts are unaware of the
missing information that might be relevant to the system.

Based on the perspectives described before, the authors identified three
challenges to requirement elicitation. The first one is to identifying
tacit knowledge (unknown knowns), even in the case the analyst presume
it exist (known unknowns). The second challenge is that the analyst
should know what is relevant and should be articulated. And the last one
is the articulation of the knowledge. Thus, in the worst case scenario
the goal is go from the unknown unknowns perspective to known knowns.

Finally, the authors suggest four research directions to deal with the
unknowness problem. Starting with the known unknown in which the analyst
should challenging assumptions, implications, and relaxing domain
constrains in order to increase the probability to elicit the unaware
knowledge of the stakeholders. In the same direction, in the design
discovery, which is a variant of the latter case, the analyst should
deal with the the following statement ``I'll know what I want when I see
it''. Thus, there is a need of prototyping or simulations as part of the
elicitation process. In the case of the unknown knowns in which the
analyst can elicit tacit knowledge by taking into account the background
(political, cultural, emotional) of stakeholders, and use that
background as an ``emotional guidance'' in order to draw out the
relevant information. Finally, in the case of unknown unknowns the
authors suggest to main approaches. The first one is based on the
``over-the-horizon'' knowledge, in which authors proposed a
socio-technical approach based on social media and e-communities. And
the other one is using counter examples that can challenges the
boundaries of the idea under development.

\hypertarget{using-argumentation-to-explain-ambiguity-in-requirements-elicitation-interviews-elrakaiby2017using}{%
\subsection{Using Argumentation to Explain Ambiguity in Requirements
Elicitation Interviews
{[}3{]}}\label{using-argumentation-to-explain-ambiguity-in-requirements-elicitation-interviews-elrakaiby2017using}}

One of the major causes of ambiguities in elicitation interviews is the
presence of tacit knowledge. In some cases ambiguities can't be always
be explained as separated term, sometimes it is required to have a
context. Even under these circumstances, the analyst must be able to
identify and alleviate them in order to elicit relevant information of
the system. For this purpose it is necessary to provide the analyst with
proper tools. In this paper Yehia Elrakaiby et al.~proposed a
theoretical framework to overcome ambiguity during interviews in the
elicitation phase. The framework is based on the ``Argumentation
theory''.

In that sense, Elrakaiby et al.~focus on one type of ambiguities, the
``acceptance unclarity''. An acceptance unclarity occurs every time the
analyst is able to assign an interpretation or meaning to the speech
fragment of the stakeholder, the interpretation matches the intended
meaning of the stakeholder, but the interpretation is not acceptable or
justified. It could be either because it seems to be inaccurate to
comprehend the problem, or analysts identify inconsistencies with their
current understanding of the problem or domain knowledge. By using
argumentation theory framework, statements and ambiguities can be
characterized as ``arguments'' and ``attacks'' respectively.

Argumentation theory models a type of human dialog based on arguments
and conclusions. It makes explicit attacks between arguments and the
argumentation flow that leads to conclusions. A basic model in this
framework is a pair \((A,D)\), where \(A\) is a set of arguments and
\(D\) is a set of attacks among those arguments. For example, a set
\(A\) is defined as \(A = \{A1,A2,A3\}\), and a possible set of attacks
could be \(D = \{(A1,A3)\}\). Which means that if A1 is realizable then
A3 can't be realizable.

In the paper, Elrakaiby et al.~models statements given by the
stakeholders, analysts domain knowledge and analysts inferences as
arguments, and ambiguities between them as attacks. For example, let say
the analyst listens the following statement \emph{the professor will
upload the task description within three days} (A1), but the analyst
know (domain knowledge) that \emph{the professor may be on a business
meeting} (A2), so the analyst think (inference) that \emph{it may be
possible that it will be take longer to upload the task description}
(A3). In this scenario the set of attacks \(D\) is given by
\(D = \{(A1,A3)\}\). Thus, since there is an attack it is possible to
ask for clarifications or details.

The theoretical framework proposed by Elrakaiby et al.~allows analysts
to detect and minimize ambiguities during elicitation interviews, while
most of the methods that focus on ambiguities analyze written texts. On
the other hand, this framework focuses in more complex ambiguities that
cannot be view as single terms.

\hypertarget{effect-of-domain-knowledge-on-elicitation-effectiveness-an-internally-replicated-controlled-experiment-aranda2016effect}{%
\subsection{Effect of Domain Knowledge on Elicitation Effectiveness: An
Internally Replicated Controlled Experiment
{[}4{]}}\label{effect-of-domain-knowledge-on-elicitation-effectiveness-an-internally-replicated-controlled-experiment-aranda2016effect}}

The effectiveness of elicitation interviews may be influenced by analyst
skills or characteristics. In these high intensive oral communication
scenario the analyst must be able to draw out relevant information and
needs from the stakeholders. It has been reported that the effectiveness
of the interviews has a direct relation with the domain knowledge of the
analyst. However, there are also studies suggesting that in some cases
the domain knowledge have negatives effects in the effectiveness of
interviews.

In this paper Aranda et al.~studied the influence of the analyst domain
knowledge on the effectiveness of elicitation interviews. The main
question they tried to answer was

\begin{quote}
\emph{Does analyst domain knowledge influence (either positively or
negatively) the effectiveness of the requirements elicitation activity?}
\end{quote}

For this purpose, the authors performed the study in two stages. They
performed an initial baseline experiment with two domain problems, and
then they performed an internal replication with two other domain
problems. Furthermore, the authors divided the elicitation process in
two phases. The elicitation phase which is the actual interview with the
stakeholder, and the reporting phase in which the analyst understands
and documents the information gathered in the elicitation phase.

One remark of the study is that the authors chose students because of
their lack of experience in elicitation interviews, isolation of the
domain knowledge, and to analyze, in the internal replication, the
influence of the training in requirements engineering in elicitation
interviews. In the study participated post-graduated students as
interviewers and two professors as interviewees. The students should
make open interviews and elicit the information afterwards. Moreover,
for each domain problem the students were separated in two groups based
on their level of domain knowledge, \emph{domain-aware} and
\emph{domain-ignorant}. On the other hand, the effectiveness of the
elicitation's was based on the comparison between the number of
concepts, processes and requirements elicited by the students and the
benchmarks.

The results of the baseline experiment suggest that the domain knowledge
of the analysts has no significant influence in the effectiveness of the
elicitation interviews. However, the results also suggest that the
domain knowledge of the interviewees has statistically significant
influence. Supplementary, the results of the internal replication also
suggest that the domain knowledge of the stakeholders is more relevant
than the analysts', in term of effectiveness of the elicitation
interviews. Nevertheless, these results also show that the positive
effects of the training in requirements engineering of the interviewers,
with these effects being as relevant as interviewees' domain knowledge.

\hypertarget{conclusions}{%
\section{Conclusions}\label{conclusions}}

I need some conclusions

\hypertarget{what-ill-do}{%
\section{What i'll do}\label{what-ill-do}}

Explain tacit knowledge explain domain knowledge Explain acceptance
unclarity Explain Dung's framework Explain ASPIC+ Inconsistencies It's
highly influenced by the domain knowledge of the analyst

\newpage

\hypertarget{references}{%
\section*{References}\label{references}}
\addcontentsline{toc}{section}{References}

\hypertarget{refs}{}
\leavevmode\hypertarget{ref-lauesen2001softwarereq}{}%
{[}1{]} S. Lauesen, \emph{Software requirements: Styles and techniques},
1st ed. Pearson Education, 2001.

\leavevmode\hypertarget{ref-sutcliffe2013requirements}{}%
{[}2{]} A. Sutcliffe and P. Sawyer, ``Requirements elicitation: Towards
the unknown unknowns,'' in \emph{Requirements engineering conference
(re), 2013 21st ieee international}, 2013, pp. 92--104.

\leavevmode\hypertarget{ref-elrakaiby2017using}{}%
{[}3{]} Y. Elrakaiby, A. Ferrari, P. Spoletini, S. Gnesi, and B.
Nuseibeh, ``Using argumentation to explain ambiguity in requirements
elicitation interviews,'' in \emph{Requirements engineering conference
(re), 2017 ieee 25th international}, 2017, pp. 51--60.

\leavevmode\hypertarget{ref-aranda2016effect}{}%
{[}4{]} A. M. Aranda, O. Dieste, and N. Juristo, ``Effect of domain
knowledge on elicitation effectiveness: An internally replicated
controlled experiment,'' \emph{IEEE Transactions on Software
Engineering}, vol. 42, no. 5, pp. 427--451, 2016.

\end{document}
