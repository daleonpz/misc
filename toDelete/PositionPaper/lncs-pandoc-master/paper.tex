\documentclass[]{llncs}

\usepackage[T1]{fontenc}
\usepackage{lmodern}
\usepackage{amssymb,amsmath}
\usepackage{ifxetex,ifluatex}
\usepackage{fixltx2e} % provides \textsubscript
% use upquote if available, for straight quotes in verbatim environments

% lncs 
\usepackage{makeidx}
\institute{FH Dortmund, \\ \texttt{}}


\IfFileExists{upquote.sty}{\usepackage{upquote}}{}
\ifnum 0\ifxetex 1\fi\ifluatex 1\fi=0 % if pdftex
  \usepackage[utf8]{inputenc}
\else % if luatex or xelatex
  \ifxetex
    \usepackage{mathspec}
    \usepackage{xltxtra,xunicode}
  \else
    \usepackage{fontspec}
  \fi
  \defaultfontfeatures{Mapping=tex-text,Scale=MatchLowercase}
  \newcommand{\euro}{€}
\fi
% use microtype if available
\IfFileExists{microtype.sty}{\usepackage{microtype}}{}
\ifxetex
  \usepackage[setpagesize=false, % page size defined by xetex
              unicode=false, % unicode breaks when used with xetex
              xetex]{hyperref}
\else
  \usepackage[unicode=true]{hyperref}
\fi
\hypersetup{breaklinks=true,
            bookmarks=true,
            pdfauthor={Daniel Paredes},
            pdftitle={The power of dark silicon},
            colorlinks=true,
            citecolor=blue,
            urlcolor=blue,
            linkcolor=magenta,
            pdfborder={0 0 0}}
\urlstyle{same}  % don't use monospace font for urls
\setlength{\parindent}{0pt}
\setlength{\parskip}{6pt plus 2pt minus 1pt}
\setlength{\emergencystretch}{3em}  % prevent overfull lines
\setcounter{secnumdepth}{5}

\title{The power of dark silicon}
\author{Daniel Paredes}
\date{}

\begin{document}
\maketitle
\begin{abstract}
Sit amet mauris. Curabitur a quam. Aliquam neque. Nam nunc nunc, lacinia
sed, varius quis, iaculis eget, ante. Nulla dictum justo eu lacus.
Phasellus sit amet quam. Nullam sodales. Cras non magna eu est
consectetuer faucibus. Donec tempor lobortis turpis. Sed tellus velit,
ullamcorper ac, fringilla vitae, sodales nec, purus. Morbi aliquet risus
in mi.

% lncs keyword extension 
\keywords{hope, luke, ewoks}

\end{abstract}

\hypertarget{introduction}{%
\section{Introduction}\label{introduction}}

I need an introduction\ldots{}

\hypertarget{using-argumentation-to-explain-ambiguity-in-requirements-elicitation-interviews}{%
\section{Using Argumentation to Explain Ambiguity in Requirements
Elicitation
Interviews}\label{using-argumentation-to-explain-ambiguity-in-requirements-elicitation-interviews}}

One of the major causes of ambiguities in elicitation interviews is the
prescense of tacit knowlegde. In some cases ambiguities can't be always
be explained as separated term, sometimes it is required to have a
context. Even under these circunstances, the analyst must be able to
identify and alleviate them in order to elicitate relevant information
of the system. For this purpose it is necessary to provide the analyst
with proper tools. In this paper Yehia Elrakaiby et al. {[}1{]} proposed
a theorical framework to overcome ambiguity during interviews in the
elicitation phase. The framework is based on the ``Argumentation
theory''.

In that sense, Elrakaiby et al.~focus on one type of ambiguities, the
``acceptance unclarity''. An acceptance unclarity occurs everytime the
analyst is able to assign an interpretation or meaning to the speech
fragment of the stakeholder, the interpretation matches the intended
meaning of the stakeholder, but the interpretation is not acceptable or
justified. It could be either because it seems to be inaccurate to
comprehend the problem, or analysts identify inconsistencies with their
current understanding of the problem or domain knowlegde. By using
argumentation theory framework, statements and ambiguities can be
characterized as ``arguments'' and ``attacks'' respectevely.

Argumentation theory models a type of human dialog based on arguments
and conclusions. It makes explicit attacks between arguments and the
argumentation flow that leads to conclusions. A basic model in this
framework is a pair \((A,D)\), where \(A\) is a set of arguments and
\(D\) is a set of attacks among those arguments. For example, a set
\(A\) is defined as \(A = \{A1,A2,A3\}\), and a possible set of attacks
could be \(D = \{(A1,A3)\}\). Which means that if A1 is realizable then
A3 can't be realizable.

In the paper, Elrakaiby et al.~models statements given by the
stakeholders, analysts domain knowlegde and analysts inferences as
arguments, and ambiguities between them as attacks. For example, let say
the analyst listens the following statement \emph{the professor will
upload the task description within three days} (A1), but the analyst
know (domain knowlegde) that \emph{the professor may be on a business
meeting} (A2), so the analyst think (inference) that \emph{it may be
possible that it will be take longer to upload the task description}
(A3). In this scenario the set of attacks \(D\) is given by
\(D = \{(A1,A3)\}\). Thus, since there is an attack it is possible to
ask for clarifications or details.

The theorical framework proposed by Elrakaiby et al.~allows analysts to
detect and minimize ambiguities during elicitation interviews, while
most of the methods that focus on ambiguities analyze written texts. On
the other hand, this framework focuses in more complex ambiguities that
cannot be view as single terms.

\hypertarget{effect-of-domain-knowledge-on-elicitation-effectiveness-an-internally-replicated-controlled-experiment}{%
\section{Effect of Domain Knowledge on Elicitation Effectiveness: An
Internally Replicated Controlled
Experiment}\label{effect-of-domain-knowledge-on-elicitation-effectiveness-an-internally-replicated-controlled-experiment}}

The effectiveness of elicitation interviews may be influenced by analyst
skills or characteristics. In these high intensive oral communication
scenario the analyst must be able to draw out relevant information and
needs from the stakeholders. It has been reported that the effectiveness
of the interviews has a direct relation with the domain knowlegde of the
analyst. However, there are also studies suggesting that in somecases
the domain knowlegde have negatives effects in the effectiveness of
interviews.

In this paper {[}2{]}, Aranda et al.~studied the influence of the
analyst domain knowlegde on the effectiveness of elicitation interviews.
The main question they tried to answer was

\emph{Does analyst domain knowledge influence (either positively or
negatively) the effectiveness of the requirements elicitation activity?}

For this purpose, the authors performed the study in two stages. They
performed an initial baseline experiment with two domain problems, and
then they performed an internal replication with two other domain
problems. Furthermore, the authors divided the elicitation process in
two phases. The elication phase which is the actual interview with the
stakeholder, and the reporting phase in which the analyst understands
and documents the information gathered in the elication phase.

One remark of the study is that the authors chose students because of
their lack of experience in eliciation interviews, isolation of the
domain knowledge, and to analyze, in the internal replication, the
influence of the training in requirements engineering in eliciation
interviews. In the study participated post-graduated students as
interviewers and two professors as interviewees. The students should
make open interviews and elicitate the information afterwards. Moreover,
for each domain problem the students were separeted in two groups based
on their level of domain knowlegde, \emph{domain-aware} and
\emph{domain-ignorant}. On the other hand, the effectiveness of the
elicitations was based on the comparison between the number of concepts,
processes and requirements elicitated by the students and the
benchmarks.

The results of the baseline experiment suggest that the domain knowlegde
of the analysts has no significant influence in the effectiveness of the
elicitation interviews. However, the results also suggest that the
domain knowlegde of the interviewees has statistically significant
influence. Supplementary, the results of the internal replication also
suggest that the domain knowlegde of the stakeholders is more relevant
than the analysts', in term of effectiveness of the elicitation
interviews. Nevertheless, these results also show that the possitive
effects of the training in requirements engineering of the interviewers,
with these effects beiang as relevant as interviwees' domain knowlegde.

\hypertarget{requirements-elicitation-towards-the-unknown-unknowns}{%
\section{Requirements elicitation: Towards the Unknown
Unknowns}\label{requirements-elicitation-towards-the-unknown-unknowns}}

\begin{center}\rule{0.5\linewidth}{\linethickness}\end{center}

Abstract---Requirements elicitation research is reviewed using a
framework categorising the relative `knowness' of requirements
specification and Common Ground discourse theory. The main contribution
of this survey is to review requirements elicitation from the
perspective of this framework and propose a road map of research to
tackle outstanding elicitation problems involving tacit knowledge. The
paper concludes with remarks on the possibility of elicitation tackling
the most difficult question of `unknown unknown' requirements.

However, elicitation still remains problematic; missing or mistaken
requirements still delay projects and cause cost overruns in this paper
we set out to explore the field from the perspective made famous from
Donald Rumsfeld's quote of the ``known knowns, the known unknowns, and
unknown unknowns''. This perspective poses challenges to requirements
elicitation since it probes the boundaries of knowledge and who
possesses it, thereby creating a useful stress test for requirements
elicitation techniques, methods and tools.

Elicitation review framework (ERF). Tacit knowledge taxonomy based on
the properties of \emph{expressible} (know knowledge),
\emph{articulated} (documented domain knowledge), \emph{accesible} (need
memory recall), \emph{relevant} to the project: • Known knowns:
expressible, articulated, and relevant. • Known unknowns: not
expressible or articulated, but accessible and potentially relevant. •
Unknown knowns: potentially accessible but not articulated. • Unknown
unknowns: not expressible, articulated or accessible but still
potentially relevant.

From the perspective of the analyst/ stakeholder: - Known knowns: not a
problem - known unknowns: the analyst has a domain knowlegde but the
stakeholder may forgot or be unaware - Unknown knowns: the stakeholder
holds a knowlegde but it is not articulated, so the analyst has to
discover this tacit knowlegde and elicitate it. Once the stakeholder
shows a glimpse of his knowlegde the analyst realize that there is
something and the problem into known unknowns. - Unknown unknowns: the
analist and the stakeholder are unaware of what's missing, but it has
the potential to be relevant information.

The Tacit Knowledge Framework {[}14{]} poses three challenges: (i)
Identifying tacit knowledge: the unknown knowns; even when the analyst
suspects they exist (known unknowns), making tacit knowledge accessible
may not be easy. (ii) Knowing what is relevant and should be articulated
from the analyst's perspective: the necessary detail problem. (iii)
Articulating the knowledge where it is needed, in the correct context,
so it can be understood by all stakeholders.

The first step to solve the problem of unknown unknowns is to assume
that may exist a missing knowledge, and being prepared to invest in
order to find it. The goal is to convert the unknown unknowns into known
knowns.

Common ground theory explains how meaning is constructed by conversation
and action, which progresses towards a mutually agreed goal.

Action Ladder and Project: Meaning in conversations has different
layers: the surface of explicit expression, tacit knowlegde and
linguistic interpretation. Setting: where the conversation takes place
(time, location, domain knowledge) Arena: shared knowledge about the
culture, norms, history and assumptions which allow dialogue between
people to be interpreted in their context.

For our purposes, Common Ground contributes `tools for thought' which
can address the unknowns problems. For example the Arena suggests
questions to discover more detail of the users' background which may
uncover unknown concerns (viz cultural, political issues);

For many techniques the ability to detect the known unknowns depends on
the analyst's plan and the sampling strategy. In the case of interviews
the ability to dectect unknown unknowns depends on the follow-up
questions and sample size. Also interviews since they depends on natural
language communication there is always the possibility of ambiguous
interpretations.

Interviews approach natural human conversation so they support the
dynamic construction of understanding and exploration of the Arena and
Setting of the dialogue by explicit questions and observation. However,
interviews are weaker for reflection(reflection as an intentional,
active progression of learning), where the analyst has to rely on notes
and recordings. Furthermore, eliciting tacit knowledge depends on the
analysts' questioning skills interaction. Hence a combination of
techniques rather than structured interviews per se is probably the most
effective approach. .

There are four research directions to push the boundaries of the
unknown: Unknown knowns. The problem for the analyst is discovering what
the stakeholder knows but does not articulate. Sensitivity to political
issues, user values and emotions needs to be researched in depth to
provide `emotional intelligence' guidance for analysts so they can
anticipate these unknowns and elicit sensitive tacit knowledge. The
Common Ground quest is to be more sensitive to the stakeholder's
Setting, feelings, norms and culture

Known unknowns. In this case the analyst has some awareness of the
necessary knowledge, so an agenda for elicitation can be set. Most
techniques involve exploring the implications of the system-domain
boundary. Challenging assumptions, reasoning about the implications of
obstacles, and relaxing domain constraints need further research towards
not only hard influences, but also soft, probabilistic implications.

Design discovery. This is a variant of known unknowns where the
challenge is to solve the ``I'll know what I want when I see it''
problem. Much progress has been made in this area with prototypes,
storyboards and mock-ups; however, simulations and virtual worlds may
have further potential.

Unknown unknowns. Two approaches could address elicitation of
`over-the-horizon' knowledge. First, creative RE, which is already
established {[}55{]}, needs to be integrated with social media, so that
collaborative creative RE is empowered. This could involve designing
socio- technical elicitation systems, as e-communities, communities of
practice with global distributions for the increasing number of Internet
applications. Secondly, the use of analogies and examples, and also
counter examples, can challenge the boundaries of the possible to
develop new design ideas.

Known unknowns imply the need for exploration of the problem and
solution spaces; (e.g.) using a combination of obstacle anaylsis and
prototyping.

\hypertarget{what-ill-do}{%
\section{What i'll do}\label{what-ill-do}}

Explain acceptance unclarity Explain Dung's framework Explain ASPIC+
Inconsistencies It's higly influenced by the domain knowledge of the
analyst

\hypertarget{my-idea}{%
\subsection{My idea}\label{my-idea}}

knowledge domain has influence analyst domain knowledge is statiscally
relevant analyst training is even more relevant influence of tacit
knowlegde in interviews

ambiguity in interviews - focus on interviewer must have domain
knowledge what i want to do is : possible beneficts in requirement
elicitaiton is usings two techniques based on the domain knowledge of
the analyst, sometimes it is good to have an expert and an ignorant of a
topic .

limitations of the technique based on domain knowlegde and tacit
knowlegde. how types of tacit knowlegde may influence in the proposal
theory how domain knowlegde may influence in the proposal theory

As an intro: analyst domain knowledge is statiscally relevant analyst
training is even more relevant focus on the technique from the
perspective of knowlegde and technique future of that technique to solve
more complex problems, from the prospective of knowlegde and technique

\newpage

\hypertarget{references}{%
\section*{References}\label{references}}
\addcontentsline{toc}{section}{References}

\hypertarget{refs}{}
\leavevmode\hypertarget{ref-elrakaiby2017using}{}%
{[}1{]} Y. Elrakaiby, A. Ferrari, P. Spoletini, S. Gnesi, and B.
Nuseibeh, ``Using argumentation to explain ambiguity in requirements
elicitation interviews,'' in \emph{Requirements engineering conference
(re), 2017 ieee 25th international}, 2017, pp. 51--60.

\leavevmode\hypertarget{ref-aranda2016effect}{}%
{[}2{]} A. M. Aranda, O. Dieste, and N. Juristo, ``Effect of domain
knowledge on elicitation effectiveness: An internally replicated
controlled experiment,'' \emph{IEEE Transactions on Software
Engineering}, vol. 42, no. 5, pp. 427--451, 2016.

\end{document}
