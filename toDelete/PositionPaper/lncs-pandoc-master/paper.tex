\documentclass[]{llncs}

\usepackage[T1]{fontenc}
\usepackage{lmodern}
\usepackage{amssymb,amsmath}
\usepackage{ifxetex,ifluatex}
\usepackage{fixltx2e} % provides \textsubscript
% use upquote if available, for straight quotes in verbatim environments

% lncs 
\usepackage{makeidx}
\institute{FH Dortmund, \\ \texttt{}}


\IfFileExists{upquote.sty}{\usepackage{upquote}}{}
\ifnum 0\ifxetex 1\fi\ifluatex 1\fi=0 % if pdftex
  \usepackage[utf8]{inputenc}
\else % if luatex or xelatex
  \ifxetex
    \usepackage{mathspec}
    \usepackage{xltxtra,xunicode}
  \else
    \usepackage{fontspec}
  \fi
  \defaultfontfeatures{Mapping=tex-text,Scale=MatchLowercase}
  \newcommand{\euro}{€}
\fi
% use microtype if available
\IfFileExists{microtype.sty}{\usepackage{microtype}}{}
\ifxetex
  \usepackage[setpagesize=false, % page size defined by xetex
              unicode=false, % unicode breaks when used with xetex
              xetex]{hyperref}
\else
  \usepackage[unicode=true]{hyperref}
\fi
\hypersetup{breaklinks=true,
            bookmarks=true,
            pdfauthor={Daniel Paredes},
            pdftitle={The power of dark silicon},
            colorlinks=true,
            citecolor=blue,
            urlcolor=blue,
            linkcolor=magenta,
            pdfborder={0 0 0}}
\urlstyle{same}  % don't use monospace font for urls
\setlength{\parindent}{0pt}
\setlength{\parskip}{6pt plus 2pt minus 1pt}
\setlength{\emergencystretch}{3em}  % prevent overfull lines
\setcounter{secnumdepth}{5}

\title{The power of dark silicon}
\author{Daniel Paredes}
\date{}

\begin{document}
\maketitle
\begin{abstract}
Sit amet mauris. Curabitur a quam. Aliquam neque. Nam nunc nunc, lacinia
sed, varius quis, iaculis eget, ante. Nulla dictum justo eu lacus.
Phasellus sit amet quam. Nullam sodales. Cras non magna eu est
consectetuer faucibus. Donec tempor lobortis turpis. Sed tellus velit,
ullamcorper ac, fringilla vitae, sodales nec, purus. Morbi aliquet risus
in mi.

% lncs keyword extension 
\keywords{hope, luke, ewoks}

\end{abstract}

\hypertarget{introduction}{%
\section{Introduction}\label{introduction}}

I need an introduction\ldots{}

\hypertarget{using-argumentation-to-explain-ambiguity-in-requirements-elicitation-interviews}{%
\section{Using Argumentation to Explain Ambiguity in Requirements
Elicitation
Interviews}\label{using-argumentation-to-explain-ambiguity-in-requirements-elicitation-interviews}}

One of the major causes of ambiguities in elicitation interviews is the
prescense of tacit knowlegde. In some cases ambiguities can't be always
be explained as separated term, sometimes it is required to have a
context. Even under these circunstances, the analyst must be able to
identify and alleviate them in order to elicitate relevant information
of the system. For this purpose it is necessary to provide the analyst
with proper tools. In this paper Yehia Elrakaiby et al. {[}1{]} proposed
a theorical framework to overcome ambiguity during interviews in the
elicitation phase. The framework is based on the ``Argumentation
theory''.

In that sense, Elrakaiby et al.~focus on one type of ambiguities, the
``acceptance unclarity''. An acceptance unclarity occurs everytime the
analyst is able to assign an interpretation or meaning to the speech
fragment of the stakeholder, the interpretation matches the intended
meaning of the stakeholder, but the interpretation is not acceptable or
justified. It could be either because it seems to be inaccurate to
comprehend the problem, or analysts identify inconsistencies with their
current understanding of the problem or domain knowlegde. By using
argumentation theory framework, statements and ambiguities can be
characterized as ``arguments'' and ``attacks'' respectevely.

Argumentation theory models a type of human dialog based on arguments
and conclusions. It makes explicit attacks between arguments and the
argumentation flow that leads to conclusions. A basic model in this
framework is a pair \((A,D)\), where \(A\) is a set of arguments and
\(D\) is a set of attacks among those arguments. For example, a set
\(A\) is defined as \(A = \{A1,A2,A3\}\), and a possible set of attacks
could be \(D = \{(A1,A3)\}\). Which means that if A1 is realizable then
A3 can't be realizable.

In the paper, Elrakaiby et al.~models statements given by the
stakeholders, analysts domain knowlegde and analysts inferences as
arguments, and ambiguities between them as attacks. For example, let say
the analyst listens the following statement \emph{the professor will
upload the task description within three days} (A1), but the analyst
know (domain knowlegde) that \emph{the professor may be on a business
meeting} (A2), so the analyst think (inference) that \emph{it may be
possible that it will be take longer to upload the task description}
(A3). In this scenario the set of attacks \(D\) is given by
\(D = \{(A1,A3)\}\). Thus, since there is an attack it is possible to
ask for clarifications or details.

The theorical framework proposed by Elrakaiby et al.~allows analysts to
detect and minimize ambiguities during elicitation interviews, while
most of the methods that focus on ambiguities analyze written texts. On
the other hand, this framework focuses in more complex ambiguities that
cannot be view as single terms.

\hypertarget{conclusion}{%
\subsection{Conclusion}\label{conclusion}}

\begin{itemize}
\tightlist
\item
  oral communication (they think it is more important than written
  texts)
\end{itemize}

\hypertarget{final-words}{%
\section{Final Words}\label{final-words}}

Nec, sodales vitae, vehicula eget, ipsum. Sed nec tortor. Aenean
malesuada. Nunc convallis, massa eu vestibulum commodo, quam mauris
interdum arcu, at pellentesque diam metus ut nulla. Vestibulum eu dolor
sit amet lacus varius fermentum. Morbi dolor enim, pulvinar eget,
lobortis ac, fringilla ac, turpis. Duis ac erat. Etiam consequat.
Integer sed est eu elit pellentesque dapibus. Duis venenatis magna
feugiat nisi. Vestibulum et turpis. Maecenas a enim. Suspendisse
ultricies ornare justo. Fusce sit amet nisi sed arcu condimentum
venenatis. Vivamus dui. Nunc accumsan, quam a fermentum mattis, magna
sapien iaculis pede, at porttitor quam odio at est.

\hypertarget{what-i-ll-focus}{%
\section{What i ll focus}\label{what-i-ll-focus}}

\begin{itemize}
\tightlist
\item
  Explain acceptance unclarity
\item
  Explain Dung's framework
\item
  Explain ASPIC+
\item
  Inconsistencies
\item
  It's higly influenced by the domain knowledge of the analyst
\end{itemize}

\hypertarget{my-idea}{%
\subsection{My idea}\label{my-idea}}

\begin{itemize}
\item
  knowledge domain has influence
\item
  analyst domain knowledge is statiscally relevant
\item
  analyst training is even more relevant
\item
  influence of tacit knowlegde in interviews
\item
  ambiguity in interviews - focus on interviewer must have domain
  knowledge
\item
  what i want to do is : possible beneficts in requirement elicitaiton
  is usings two techniques based on the domain knowledge of the analyst,
  sometimes it is good to have an expert and an ignorant of a topic .
\item
  limitations of the technique based on domain knowlegde and tacit
  knowlegde.
\item
  how types of tacit knowlegde may influence in the proposal theory
\item
  how domain knowlegde may influence in the proposal theory
\item
  As an intro:

  \begin{itemize}
  \tightlist
  \item
    analyst domain knowledge is statiscally relevant
  \item
    analyst training is even more relevant
  \end{itemize}
\item
  focus on the technique from the perspective of knowlegde and technique
\item
  future of that technique to solve more complex problems, from the
  prospective of knowlegde and technique
\end{itemize}

\newpage

\hypertarget{references}{%
\section*{References}\label{references}}
\addcontentsline{toc}{section}{References}

\hypertarget{refs}{}
\leavevmode\hypertarget{ref-elrakaiby2017using}{}%
{[}1{]} Y. Elrakaiby, A. Ferrari, P. Spoletini, S. Gnesi, and B.
Nuseibeh, ``Using argumentation to explain ambiguity in requirements
elicitation interviews,'' in \emph{Requirements engineering conference
(re), 2017 ieee 25th international}, 2017, pp. 51--60.

\end{document}
